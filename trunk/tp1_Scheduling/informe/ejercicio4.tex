
\begin{section}{Ejercicio 4}

En los casos de los test "ts*", las colas nunca se encuentran vacías
Análisis FCFS:
	Test UNO:
		Como la cola NUNCA está vacía, siempre puede estar ejecutando un proceso, como el tiempo de cada proceso en ts1 es creciente, eso ayuda para bajar el tiempo de espera.
	-------------------------------------
	Resultados analisis del scheduling producido:
	tiempo de espera promedio: 87
	-------------------------------------
dos:
	Con una diferencia de 5 "nexts" entre cada una entran los procesos, cada uno entran con el tiempo para correr mayor q el siguiente, x lo tanto el wtime aumenta
	-------------------------------------
	Resultados analisis del scheduling producido:
	tiempo de espera promedio: 192
	-------------------------------------
tres:
	Todos comienzan (ready) en el momento cero. decrecen dos, uno x, crecen dos
	-------------------------------------
	Resultados analisis del scheduling producido:
	tiempo de espera promedio: 127
	-------------------------------------
cuatro:
	Cuatro inician en cero, cuatro en el tiempo 100, x lo tanto claro el wtime de esos cuatro posiblemente sea menos, mismo tiempo de cada task
	-------------------------------------
	Resultados analisis del scheduling producido:
	tiempo de espera promedio: 77
	-------------------------------------
cinco:
	Server, todos inician en tiempo cero, inicia bd, server, testeo, acepta clientes, apaga servidor, apaga bd
	cada cliente tiene mucha diferencia en cantidad de tiempo q usa
	-------------------------------------
	Resultados analisis del scheduling producido:
	tiempo de espera promedio: 144
	-------------------------------------
seis:
	mismo server con solo dos clientes mas de 10 y de 5 de gasto, y aumenta bastante mas el wtime.
	-------------------------------------
	Resultados analisis del scheduling producido:
	tiempo de espera promedio: 176
	-------------------------------------


\end{section}

