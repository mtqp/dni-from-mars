\documentclass[12pt,titlepage]{article}
\usepackage[spanish]{babel}
\usepackage[utf8]{inputenc}

\newcommand{\func}[1]{\texttt{#1}}
\newcommand{\cmd}[1]{\texttt{#1}}
\newcommand{\code}[1]{\texttt{#1}}
\newcommand{\kw}[1]{{\em #1}}

\title{{\sc\normalsize Sistemas Operativos}\\{\bf Taller de Syscalls}}
\author{Grupo 20\vspace*{3em} \\ 
\begin{tabular}{lcr}
Daniel Grosso & 694/08 & dgrosso@gmail.com\\
Mariano De Sousa Bispo & 389/08 & marian\_sabianaa@hotmail.com \\
\end{tabular}}
\date{\vspace*{3em} \normalsize{Septiembre 2010}}

\begin{document}
\begin{titlepage}
\maketitle
\end{titlepage}

\begin{section}{Explicación de \cmd{mister}}
	El ejecutable \cmd{mister} al inicio genera dos \kw{pipes} (que llamaremos \code{p1} y \code{p2}) y un proceso hijo. El comportamiento del proceso padre y el proceso hijo se describe a continuación:
	
	\vspace*{1.5em}
	\noindent{\bf Proceso padre}
	\begin{itemize}
		\item Establece \code{p1} como \kw{pipe} de sólo escritura y \code{p2} como \kw{pipe} de sólo lectura. 
		\item Escribe la palabra ``vaca'' en \code{p1} y espera a que el proceso hijo envíe una respuesta a través de \code{p2}. 
		\item Repite el mismo procedimiento con dos frases distintas.
		\item Escribe la palabra ``chau'' en \code{p1} y espera a que el proceso hijo envíe una respuesta a través de \code{p2}. 
		\item Cierra los \kw{pipes}.
		\item Termina.
	\end{itemize}
	\noindent{\bf Proceso hijo}
	\begin{itemize}
		\item Establece \code{p1} como \kw{pipe} de sólo lectura y \code{p2} como \kw{pipe} de sólo escritura. 
		\item Espera a recibir un \kw{string} en \code{p1} y escribe en \code{p2} la longitud del \kw{string}.
		\item Repite el paso anterior hasta que se cierra el \code{pipe} del que se está leyendo
		\item Cierra los \kw{pipes}.
		\item Termina.
	\end{itemize}
\end{section}

%\begin{section}{\cmd{nofork}}

%\end{section}

\end{document}
