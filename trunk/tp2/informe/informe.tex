\documentclass[12pt,titlepage]{article}
\usepackage[spanish]{babel}
\usepackage[utf8]{inputenc}

\newcommand{\func}[1]{\texttt{#1}}
\newcommand{\cmd}[1]{\texttt{#1}}
\newcommand{\code}[1]{\texttt{#1}}
\newcommand{\kw}[1]{{\em #1}}

\title{{\sc\normalsize Sistemas Operativos}\\{\bf Taller de Syscalls}}
\author{Grupo 20\vspace*{3em} \\ 
\begin{tabular}{lcr}
Daniel Grosso & 694/08 & dgrosso@gmail.com\\
Mariano De Sousa Bispo & 389/08 & marian\_sabianaa@hotmail.com \\
\end{tabular}}
\date{\vspace*{3em} \normalsize{Septiembre 2010}}

\begin{document}
\begin{titlepage}
\maketitle
\end{titlepage}

\begin{section}{Explicación de \cmd{mister}}
	El ejecutable \cmd{mister} al inicio genera dos \kw{pipes} (que llamaremos \code{p1} y \code{p2}) y un proceso hijo. El comportamiento del proceso padre y el proceso hijo se describe a continuación:
	
	\vspace*{1.5em}
	\noindent{\bf Proceso padre}
	\begin{itemize}
		\item Establece \code{p1} como \kw{pipe} de sólo escritura y \code{p2} como \kw{pipe} de sólo lectura. 
		\item Escribe la palabra ``vaca'' en \code{p1} y espera a que el proceso hijo envíe una respuesta a través de \code{p2}. 
		\item Repite el mismo procedimiento con dos frases distintas.
		\item Escribe la palabra ``chau'' en \code{p1} y espera a que el proceso hijo envíe una respuesta a través de \code{p2}. 
		\item Cierra los \kw{pipes}.
		\item Termina.
	\end{itemize}
	\vspace*{1.5em}
	\noindent{\bf Proceso hijo}
	\begin{itemize}
		\item Establece \code{p1} como \kw{pipe} de sólo lectura y \code{p2} como \kw{pipe} de sólo escritura. 
		\item Espera a recibir un \kw{string} en \code{p1} y escribe en \code{p2} la longitud del \kw{string}.
		\item Repite el paso anterior hasta que el padre cierra los \kw{pipes}.
		\item Cierra los \kw{pipes}.
		\item Termina.
	\end{itemize}
\end{section}
\newpage
\begin{section}{Programa \cmd{nofork}}
	El programa \code{nofork} ejecuta el comando pasado por	parámetro, no permitiéndole a este ejecutar ninguna de las \kw{syscalls} prohibidas: \code{fork}, \code{clone}. Para ello, crea un proceso hijo mediante la instrucción \code{fork}, quien llama a la \kw{syscall} \code{ptrace} con parámetro \code{PTRACE\_TRACEME} para permitirle al padre rastrear la ejecución\footnote{\code{PTRACE\_TRACEME} hace que cualquier señal (salvo \code{SIGKILL}) enviada al proceso lo detenga y notifica al padre vía \code{wait}.}. Luego reemplaza al proceso actual (hijo) con el comando pasado por parámetro mediante la instrucción \code{execvp}. Por otra parte, el padre espera a que se produzca un cambio de estado en el proceso hijo vía \code{wait} y, mediante una llamada a \code{ptrace} con parámetro \code{PTRACE\_PEEKUSER}, obtiene el valor del registro \code{EAX}, que corresponde al número de \kw{syscall} que ejecutó el proceso hijo. Si la \kw{syscall} era \code{fork} o \code{clone}, mata al proceso con una llamada a \code{ptrace} con parámetro \code{PTRACE\_KILL}, imprime \kw{Syscall prohibida} en \code{stdout} y termina el programa. En caso contrario, se sigue la ejecución del hijo hasta la próxima \kw{syscall} a través de una llamada a \code{ptrace} con parámetro \code{PTRACE\_SYSCALL}. Si ninguna de las \kw{syscalls} prohibidas fue ejecutada, el proceso hijo se ejecuta con normalidad y, al terminar de ejecutarse, termina el programa.
\end{section}

\end{document}
