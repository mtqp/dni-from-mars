
\begin{section}{Ejercicio 6}

\begin{subsection}{Análisis de los resultados de \cmd{ts1..6}}
	Los valores de \kw{WTP} obtenidos con \cmd{MFQ} fueron de 171, 259, 217, 148, 138 y 144 unidades de tiempo respectivamente. Estos altos valores de \kw{waiting time} se deben a que los procesos son bajados de prioridad cuando se les agota el \kw{quantum} y deben esperar a que todos los procesos que tienen prioridad superior terminen para poder continuar su ejecución. En este caso, la cola de prioridad alta posee un \kw{quantum} bajo de 5 unidades de tiempo. Como la mayoría de los procesos tienen una duración mayor a ese \kw{quantum}, cuando este se les acaba son desalojados hacia la cola de prioridad media con un nuevo quantum ahora de 15 unidades. Esto implica que el tiempo de espera de cada tarea en la cola de prioridad media es alto. La gran desventaja de la cola de alta prioridad es que debido a su bajo \kw{quantum}, se produce una gran cantidad de \kw{task switch}, por lo que hay muchos tiempos muertos (2 por cada \kw{switch}). Cuando las tareas son ejecutadas en la cola de prioridad media, se ejecutan por más tiempo para compensar el tiempo de espera y con un \kw{quantum} mayor (de 15 unidades de tiempo), logrando en algunos casos terminar su ejecución. Por último, los procesos que no terminaron en prioridad media, son enviados a la cola de baja prioridad en donde se les otorga un \kw{quantum} aún mayor al de la cola de prioridad media (de 45 unidades de tiempo), pero el tiempo de espera de los procesos que se encuentran en esta cola es muy alto. Estas tareas permanecen las próximas rondas allí hasta terminar su ejecución.
\end{subsection}

\end{section}

